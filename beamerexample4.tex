\documentclass[cjk]{beamer}


\usepackage{CJK}
\usepackage{beamerthemesplit}

\begin{document}
  \begin{CJK}{GB}{kai}

    \title[ Juego Enigma]{
      PROYECTO ANDROID\\
      Juego Enigma\\
      }
    \author{Vanessa Robles, Ana Cruz, Ricardo Campuzano}
    \date{\today}

      \pgfdeclaremask{apple}{beamer-g4-mask}

       \pgfdeclareimage[interpolate=true,mask=apple,%
                 width=1.625cm,height=2cm]{apple}{beamer-g4}

    \frame{\titlepage}
    
    \section*{Temas}
    \frame{\tableofcontents}
    
    \section{Introduccion}
    \subsection{Que es Enigma}

    \frame{
      \frametitle{Que es Enigma}
      \begin{block}{}
        Enigma es un videojuego de puzzle inspirado en Oxyd  disponible para multitud de sistemas: 
      \end{block}
      \begin{itemize}
      \item<1-> UNIX como Mac OS X.
      \item<2-> Linux (mayoria de distribuciones Linux).
      \item<3-> Windows.
      \end{itemize}
    }

    \section{Mecanica}
    \subsection{Como se juega Enigma}

    \frame{
      \frametitle{Como se juega Enigma}
      \begin{block}{}
        Una canica debe moverse por el escenario y ha de golpear unos bloques para emparejar los del mismo tipo. Hay muchos objetos distribuidos por cada pantalla, pero no se dan explicaciones sobre que hacen o para que sirven, y es el jugador quien tiene que descubrirlo. En algunos niveles se manejan varias canicas a la vez.
      \end{block}



    }
  \end{CJK}
\end{document}

